\documentclass[
    a4paper, % Page size
    fontsize=11pt, % Base font size
    twoside=true, % Use different layouts for even and odd pages (in particular, if twoside=true, the margin column will be always on the outside)
	%open=any, % If twoside=true, uncomment this to force new chapters to start on any page, not only on right (odd) pages
	%chapterentrydots=true, % Uncomment to output dots from the chapter name to the page number in the table of contents
	numbers=noenddot, % Comment to output dots after chapter numbers; the most common values for this option are: enddot, noenddot and auto (see the KOMAScript documentation for an in-depth explanation)
	fontmethod=modern, % Can also use "modern" with XeLaTeX or LuaTex; "tex" is the default for PdfLaTex, and "modern" is the default for those two.
]{kaobook}

\usepackage[language=english,debug=false]{ethesis}

% Use English
\usepackage{polyglossia}
\setmainlanguage{english}
\usepackage[english=british]{csquotes}	% English quotes

% Load packages for testing
\usepackage{blindtext} % Print text without any meaning for testing purposes

% Load the bibliography package
\usepackage{kaobiblio}
\addbibresource{thesis.bib} % Bibliography file

% Load mathematical packages for theorems and related environments
\usepackage[framed=true]{kaotheorems}

% Load the package for hyperreferences
\usepackage{kaorefs}

\graphicspath{{examples/documentation/images/}{images/}} % Paths in which to look for images

\makeindex[columns=3, title=Alphabetical Index, intoc] % Make LaTeX produce the files required to compile the index

\makeglossaries % Make LaTeX produce the files required to compile the glossary
% used with \gls
\newglossaryentry{computer}{
    name=computer,
    description={is a programmable machine that receives input, stores and manipulates data, and provides output in a useful format}
}
\newglossaryentry{MRlbl}{
    name=Magnetic Resonance,
    description={is a phenomenon where atomic nuclei in a strong constant magnetic field, excited by a second oscillating magnetic field of correct frequency resonate and emit a corresponding electromagnetic signal}
}
\newglossaryentry{NMRlbl}{
    name=Nuclear Magnetic Resonance spectroscopy,
    description={is a spectroscopic technique to observe shifts in magnetic field strengths in atomic nuclei. Or in simpler terms a \href{https://blogs.bath.ac.uk/csct/2016/hidden-force-looking-machine/}{Hidden Force Looking Machine}}
}
\newglossaryentry{CW-NMRlbl}{
    name=Continous Wave Nuclear Magnetic Resonance spectroscopy,
    description={is an \gls{NMRlbl} experiment performed by slowly varying the excitation frequency and observing the resonance behaviour of nuclei inside the sample}
}
\newglossaryentry{FT-NMRlbl}{
    name=Fourier Transform Nuclear Magnetic Resonance spectroscopy,
    description={Instead of slowly varying the frequency as in \gls{CW-NMRlbl}, a single pulse of a given length with a fixed frequency is sent. According to Fourier's theory, this \enquote{smears} the frequency and excites nuclei with different resonant frequencies simultaneously}
}
\newglossaryentry{BNClbl}{
    name=Bayonet Neill-Concelman connector,
    description={is a widely used small coaxial RF connector with a quick bayonet connection designed for up to a few GHz. A \qty{50}{\ohm} and a \qty{75}{\ohm} variant exist. It is used on radio systems, oscilloscopes, test equipment and video hardware}
}
\newglossaryentry{SMAlbl}{
    name=SubMiniature version A connector,
    description={is a common small coaxial RF connector with a screw-on connection designed for high frequencies (tens of GHz) often used for WiFi antennas and handheld radio devices among others}
}
\newglossaryentry{python}{
    name=Python,
    description={is a high-level interpreted general-purpose cross-platform programming language designed by Guido van Rossum. It focuses on readability and ease of use with a \enquote{batteries included} approach. The \href{https://peps.python.org/pep-0020/}{Zen of Python} states among others: \enquote{There should be one -- and preferably only one -- obvious way to do it}. (For the seasoned \enquote{Pythoneer}: Try \enquote{\lstinline[language=Python]{import this}} and \enquote{\lstinline[language=Python]{import antigravity}})}
}
\newglossaryentry{RPlbl}{
    name=RedPitaya,
    description={is a company producing credit card sized computers including an \acrshort{fpga} which run open-source software (the hardware itself is \emph{not} open-source). In the context of the thesis, this refers to the specific board used -- a RedPitaya \href{https://redpitaya.com/sdrlab-122-16/}{SDRlab 122-16} based on a Xilinx \acrshort{soc} (Zynq 7020 with Dual ARM Cortex A9, \qty{122.88}{\mega\hertz} clock and \qty{50}{\ohm} terminated \acrshort{rf} inputs)}
}
\newglossaryentry{GNUlbl}{
    name=GNU,
    description={is an extensive collection of \gls{fs} developed within the GNU Project which can be used to form a complete operating system -- the most famous one known as Linux. Its goal is to give computer users freedom and control in their use of computers by developing software under the copyleft \acrshort{gpl}, that guarantees access to the source code of the software}
}
\newglossaryentry{fs}{
    name=Free Software,
    description={is software available with a license that allows everyone to not only run but also look at, change and redistribute it freely -- i.e. it is distributed with the source code. It is thus different from freeware and especially proprietary software}. The FSF puts it as: \enquote{Think of \enquote{free} as in \enquote{free speech} not as in \enquote{free beer}.}
}
\newglossaryentry{lcrmeter}{
    name=LCR meter,
    description={is an electronic test equipment to measure inductance (L), capacitance (C) and resistance (R)}
}
\newglossaryentry{kicad}{
    name=KiCAD,
    description={is an open source \gls{fs} schematic capture and electronic design software endorsed by CERN for open hardware development}
}

% Glossary entries (used in text with e.g. \acrfull{fpsLabel} or \acrshort{fpsLabel})
\newacronym[longplural={Frames per Second}]{fpsLabel}{FPS}{Frame per Second}
\newacronym[longplural={Tables of Contents}]{tocLabel}{TOC}{Table of Contents}

\newacronym{mr}{MR}{\gls{MRlbl}}
\newacronym{nmr}{NMR}{\gls{NMRlbl}}
\newacronym{ftnmr}{FT-NMR}{\gls{FT-NMRlbl}}
\newacronym{cwnmr}{CW-NMR}{\gls{CW-NMRlbl}}
\newacronym{sma}{SMA}{\gls{SMAlbl}}
\newacronym{bnc}{BNC}{\gls{BNClbl}}
\newacronym{rf}{RF}{radio frequency}
\newacronym{tx}{TX}{transmit}
\newacronym{rx}{RX}{receive}
\newacronym{lna}{LNA}{low-noise amplifier}
\newacronym{rp}{RP}{\gls{RPlbl}}
\newacronym{fpga}{FPGA}{Field-Programmable Gate Array}
\newacronym{soc}{SoC}{System on a Chip}
\newacronym{pa}{PA}{power amplifier}
\newacronym{fid}{FID}{Free Induction Decay}
\newacronym{linux}{Linux}{\gls{GNUlbl}/Linux}
\newacronym{gnu}{GNU}{\gls{GNUlbl}'s Not Unix!}
\newacronym{gpl}{GPL}{\gls{GNUlbl} General Public License}
\newacronym{ip}{IP}{Internet Protocol}
\newacronym{fft}{FFT}{Fast Fourier Transform}
\newacronym{dhcp}{DHCP}{Dynamic Host Configuration Protocol}
\newacronym{ssh}{SSH}{Secure Shell}
\newacronym{pep}{PEP}{Python Enhancement Proposal}
\newacronym{pip}{pip}{pip Install Packages}
\newacronym{mri}{MRI}{Magnetic Resonance Imaging}
\newacronym{mmic}{MMIC}{Monolithic Microwave Integrated Circuit}
\newacronym{dma}{DMA}{Direct Memory Access}
\newacronym{tr}{T/R}{Transmit/Receive}
\newacronym{pcb}{PCB}{Printed Circuit Board}
\newacronym{marcos}{MaRCoS}{MAgnetic Resonance COntrol System}
\newacronym{cic}{CIC}{Cascaded Integrator-Comb}
\newacronym{spdt}{SPDT}{Single Pole, Double Throw}
\newacronym{spi}{SPI}{Serial Peripheral Interface}
\newacronym{adc}{ADC}{Analogue-to-ditigal converter}
\newacronym{opamp}{OpAmp}{Operational Amplifier}
\newacronym{vna}{VNA}{Vector Network Analyser}
\newacronym{dc}{DC}{direct current}
\newacronym{sdr}{SDR}{Software Defined Radio}
\newacronym{mpn}{MPN}{Manufacturer Part Number}
\newacronym{api}{API}{Application Programming Interface}
\newacronym{bom}{BOM}{Bill Of Material} % Include the glossary definitions

\makenomenclature % Make LaTeX produce the files required to compile the nomenclature

% Reset sidenote counter at chapters
%\counterwithin*{sidenote}{chapter}

%----------------------------------------------------------------------------------------

\begin{document}

%----------------------------------------------------------------------------------------
%	BOOK INFORMATION
%----------------------------------------------------------------------------------------

\title{Building a \qty{25}{\mega\hertz} NMR Spectrometer}
\subtitle{}
\logo{\includegraphics[width=4cm]{ethlogo}}
\author{Maximilian Stabel}
\born{03.11.1995}
\citizen{Germany}
\date{\today}

\examiner{Prof.\ Roland Riek\\Prof.\ Sebastian Kozerke}
\supervisor{Dr.\ Takuya Segawa}

\degree{Master of Science ETH in Electrical Engineering and Information Technology}
\shortdegree{MSc ETH EEIT}
\idnumber{20\=/954\=/590}  % Student ID

%----------------------------------------------------------------------------------------

\frontmatter % Denotes the start of the pre-document content, uses roman numerals

%----------------------------------------------------------------------------------------
%	COLOPHON
%----------------------------------------------------------------------------------------

\lowertitleback{
	\hypersetup{hidelinks}
	\color{blankcolor}
	\footnotesize

	\begin{tblr}{@{}rX[l]@{}}
		Name                                                    & \texttt{\textbf{\jobname{}}}                                                                                                      \\
		Compiled on                                             & \textbf{\DTMnow{}}                                                                                                                \\
		\faIcon{git-alt}                                        & \texttt{\textbf{\GitShortSHA{}} (\GitRefName{})}                                                                                  \\
		Engine                                                  & \prettybanner{}                                                                                                                   \\
		\href{https://www.latex-project.org/}{\LaTeX{} Version} & \hologo{\fmtname} (\fmtversion)                                                                                                   \\
		\glossaryname{}                                         & \texttt{makeglossaries}                                                                                                           \\
		\bibname{}                                              & biblatex + \hologo{biber}                                                                                                         \\
		Generator                                               & \texttt{latexmk}                                                                                                                  \\
		Class                                                   & \href{https://sourceforge.net/projects/koma-script/}{\KOMAScriptVersion{}} + \href{https://github.com/fmarotta/kaobook/}{kaobook} \\
		Font                                                    & \textrm{\showfont}, \textsf{Sans:\ \showfont}, \texttt{Mono:\ \showfont}                                                          \\
		Math Font                                               & \mathfont, Operators:\ \mathoperatorfont                                                                                          \\
		Font Size                                               & \normalfontsize{}pt                                                                                                               \\
	\end{tblr}%
}

%----------------------------------------------------------------------------------------
%	DEDICATION
%----------------------------------------------------------------------------------------

\dedication{
	\textit{%
		I may not have gone where I intended to go,\\
		but I think I have ended up where I needed to be.\\
	}%
	--- Douglas Adams
}

%----------------------------------------------------------------------------------------
%	OUTPUT TITLE PAGE AND PREVIOUS
%----------------------------------------------------------------------------------------

% Includes colophon, dedication, ...
\maketitle

%----------------------------------------------------------------------------------------
%	PREFACE
%----------------------------------------------------------------------------------------

\import{chapters/}{abstract.tex}
\index{abstract}

%----------------------------------------------------------------------------------------
%	TABLE OF CONTENTS & LIST OF FIGURES/TABLES
%----------------------------------------------------------------------------------------

\begingroup % Local scope for the following commands

% Define the style for the TOC, LOF, and LOT
%\setstretch{1} % Uncomment to modify line spacing in the ToC
%\hypersetup{linkcolor=blue} % Uncomment to set the colour of links in the ToC
\setlength{\textheight}{230\hscale} % Manually adjust the height of the ToC pages

% Turn on compatibility mode for the etoc package
\etocstandarddisplaystyle % "toc display" as if etoc was not loaded
\etocstandardlines % "toc lines as if etoc was not loaded

\tableofcontents % Output the table of contents

\listoffigures % Output the list of figures

% Comment both of the following lines to have the LOF and the LOT on 
% different pages
\let\cleardoublepage\bigskip
\let\clearpage\bigskip

\listoftables % Output the list of tables

\listoflstlistings % Output the list of listings

\endgroup

%----------------------------------------------------------------------------------------
%	MAIN BODY
%----------------------------------------------------------------------------------------

\mainmatter % Denotes the start of the main document content, resets page numbering and uses arabic numbers
\setchapterstyle{kao} % Choose the default chapter heading style

\setchapterpreamble[u]{\margintoc}
\chapter{Introduction}
\labch{intro}

\section{Concept}
Idea/motivation
given magnet
schema of setup
Design decisions
rf switch
amplifiers

\section{Complete Setup}

\begin{circuitikz}[european]
	\ctikzset{bipoles/amp/width=0.9}
	\draw[nodes={align=center}]
	(0,0) coordinate(mid)

	% TX
	(mid) ++(0,3) node[draw, align=center, minimum height=5.5cm, minimum width=2cm](decoder){Sequence\\Decoder}
	($(decoder.east)!0.75!(decoder.north east)$) coordinate(decoderi)
	($(decoder.east)!0.75!(decoder.south east)$) coordinate(decoderq)
	($(decoder.north west)!0.5!(decoder.south west)$) coordinate(decoderin)
	(decoderin) ++(-1,0) node[left]{Pulse\\Sequence} coordinate(seq) -- (decoderin) node[inputarrow]{}
	(decoderi) node[left]{I} ++(3,0) node[mixer](mi1){}
	(decoderq) node[left]{Q} ++(3,0) node[mixer](mq1){}
	(decoderi) -- (mi1.w) node[inputarrow]{}
	(decoderq) -- (mq1.w) node[inputarrow]{}
	(decoderin -| mi1.s) coordinate (center1)
	(center1) ++(-1,0) node[oscillator](osc1){} -- (center1) -- (mi1.s) node[inputarrow,rotate=90]{}
	(osc1.n) node[above]{NCO}
	(osc1.s) node[below]{\qty{25}{MHz}}
	(center1) to[phaseshifter,>,l=90°] (mq1.n) node[inputarrow,rotate=270]{}
	(center1) ++(2,0) node[adder](add1){}
	(mi1.e) -| (add1.n) node[inputarrow,rotate=270]{}
	(mq1.e) -| (add1.s) node[inputarrow,rotate=90]{}
	(add1.e) to[dac,>] ++(2,0) to[amp,t=PA,l=Power\\Amplifier,>] ++(2,0) coordinate(tx)

	% Circulator
	(mid -| tx) node[circulator](circ){} node[below right]{passive\\TX/RX Switch}
	(tx) -| (circ.n) node[inputarrow,rotate=270]{}

	% RX
	(mid) ++(0,-3) coordinate(center_rx)
	(center_rx -| circ) coordinate(rx_out)
	(circ.s) |- (rx_out)
	(rx_out) to[amp,t=\rotatebox{180}{LNA},l=Low Noise\\Amplifier,>] ++(-2,0) to[adc,>] ++(-2,0) -- (rx_out -| add1) coordinate(rx_split)
	(rx_split) ++(0,2) coordinate(rx_upper)
	(rx_split) ++(0,-2) coordinate(rx_lower)
	(rx_upper -| mi1) node[mixer](mi2){}
	(rx_lower -| mq1) node[mixer](mq2){}
	(rx_split) |- (mi2.e) node[inputarrow,rotate=180]{}
	(rx_split) |- (mq2.e) node[inputarrow,rotate=180]{}
	(rx_split -| osc1) node[oscillator](osc2){}
	(osc2.n) node[above]{NCO}
	(osc2.s) node[below]{\qty{25}{MHz}}
	(osc2.e) -| (mi2.s) node[inputarrow,rotate=90]{}
	(osc2.e -| mq2) to[phaseshifter,>,l=90°] (mq2.n) node[inputarrow,rotate=270]{}
	(mi2.w) to[bandpass,l_=CIC,>] (mi2 -| decoderi) node[inputarrow,rotate=180]{} node[left](circ_out_i){I}
	(circ_out_i) -- (mi2 -| seq) node[left]{I} node[inputarrow,rotate=180]{}
	(mq2.w) to[bandpass,l=CIC,>] (mq2 -| decoderq) node[inputarrow,rotate=180]{} node[left](circ_out_q){Q}
	(circ_out_q) -- (mq2 -| seq) node[left]{Q} node[inputarrow,rotate=180]{}

	% Probe
	(circ.e) -- ++(2,0) node[dinantenna](probe){Probe}
	;

	% Red Pitaya
	\draw[dashed]
	($(seq)!0.5!(decoderin)$) ++(0,3) coordinate(rectangle_start)
	($(add1.e)!0.5!(tx)$) coordinate(rp_out)
	(rp_out |- mq2) ++(0,-1) coordinate(rectangle_end)
	(rectangle_start) rectangle (rectangle_end)
	(rectangle_start) node[above right]{Red Pitaya}
	;
\end{circuitikz}

\pagelayout{wide} % No margins
\addpart{Class Options, Commands and Environments}
\pagelayout{margin} % Restore margins

\input{chapters/options.tex}
\input{chapters/textnotes.tex}
\input{chapters/figsntabs.tex}
\input{chapters/references.tex}

\pagelayout{wide} % No margins
\addpart{Design and Additional Features}
\pagelayout{margin} % Restore margins

\input{chapters/layout.tex}
\input{chapters/mathematics.tex}

\appendix % From here onwards, chapters are numbered with letters, as is the appendix convention

\pagelayout{wide} % No margins
\addpart{Appendix}
\pagelayout{margin} % Restore margins

\chapter{Schematics}
\labch{schematics}
\section{Power Amplifier}
\includegraphics[angle=90,width=0.9\textwidth]{poweramp.pdf}

\section{Switch}
\labsec{switch-schematic}
\includegraphics[angle=90,width=\textwidth]{tr_switch.pdf}

\section{Low Noise Amplifier}
\labsec{lna-schematic}
\includegraphics[angle=90,width=\textwidth]{preamp.pdf}

\section{32-channel current source}
\labsec{32-channel-current-source-schematic}
\includegraphics[angle=90,width=\textwidth,page=1]{32-channel_current_source.pdf}

\includegraphics[angle=90,width=\textwidth,page=2]{32-channel_current_source.pdf}

\includegraphics[angle=90,width=\textwidth,page=3]{32-channel_current_source.pdf}

% \chapter{About this thesis}
% \labch{about}

% This thesis was heavily influenced by Jean-Luc Doumont's \enquote{English Communication for Scientists} \sidecite{doumontEnglishCommunicationScientists2010}.

% \chapter{Schematics}



\pagelayout{wide}
\setchapterstyle{kao}
\chapter{Test}
\blindtext

%----------------------------------------------------------------------------------------

\backmatter % Denotes the end of the main document content
\setchapterstyle{plain} % Output plain chapters from this point onwards

%----------------------------------------------------------------------------------------
%	BIBLIOGRAPHY
%----------------------------------------------------------------------------------------

% The bibliography needs to be compiled with biber using your LaTeX editor, or on the command line with 'biber main' from the template directory

\defbibnote{bibnote}{Here are the references in citation order.\par\bigskip} % Prepend this text to the bibliography
\printbibliography[heading=bibintoc, title=Bibliography, prenote=bibnote] % Add the bibliography heading to the ToC, set the title of the bibliography and output the bibliography note

%----------------------------------------------------------------------------------------
%	NOMENCLATURE
%----------------------------------------------------------------------------------------

% The nomenclature needs to be compiled on the command line with 'makeindex main.nlo -s nomencl.ist -o main.nls' from the template directory

\nomenclature{$c$}{Speed of light in a vacuum inertial frame}
\nomenclature{$h$}{Planck constant}

\renewcommand{\nomname}{Notation} % Rename the default 'Nomenclature'
\renewcommand{\nompreamble}{The next list describes several symbols that will be later used within the body of the document.} % Prepend this text to the nomenclature

\printnomenclature % Output the nomenclature

%----------------------------------------------------------------------------------------
%	GREEK ALPHABET
% 	Originally from https://gitlab.com/jim.hefferon/linear-algebra
%----------------------------------------------------------------------------------------

\vspace{1cm}

{\usekomafont{chapter}Greek Letters with Pronunciations} \\[2ex]
\begin{center}
	\newcommand{\pronounced}[1]{\hspace*{.2em}\small\textit{#1}}
	\begin{tabular}{l l @{\hspace*{3em}} l l}
		\toprule
		Character            & Name                            & Character              & Name                                   \\
		\midrule
		$\alpha$             & alpha \pronounced{AL-fuh}       & $\nu$                  & nu \pronounced{NEW}                    \\
		$\beta$              & beta \pronounced{BAY-tuh}       & $\xi$, $\Xi$           & xi \pronounced{KSIGH}                  \\
		$\gamma$, $\Gamma$   & gamma \pronounced{GAM-muh}      & o                      & omicron \pronounced{OM-uh-CRON}        \\
		$\delta$, $\Delta$   & delta \pronounced{DEL-tuh}      & $\pi$, $\Pi$           & pi \pronounced{PIE}                    \\
		$\epsilon$           & epsilon \pronounced{EP-suh-lon} & $\rho$                 & rho \pronounced{ROW}                   \\
		$\zeta$              & zeta \pronounced{ZAY-tuh}       & $\sigma$, $\Sigma$     & sigma \pronounced{SIG-muh}             \\
		$\eta$               & eta \pronounced{AY-tuh}         & $\tau$                 & tau \pronounced{TOW (as in cow)}       \\
		$\theta$, $\Theta$   & theta \pronounced{THAY-tuh}     & $\upsilon$, $\Upsilon$ & upsilon \pronounced{OOP-suh-LON}       \\
		$\iota$              & iota \pronounced{eye-OH-tuh}    & $\phi$, $\Phi$         & phi \pronounced{FEE, or FI (as in hi)} \\
		$\kappa$             & kappa \pronounced{KAP-uh}       & $\chi$                 & chi \pronounced{KI (as in hi)}         \\
		$\lambda$, $\Lambda$ & lambda \pronounced{LAM-duh}     & $\psi$, $\Psi$         & psi \pronounced{SIGH, or PSIGH}        \\
		$\mu$                & mu \pronounced{MEW}             & $\omega$, $\Omega$     & omega \pronounced{oh-MAY-guh}          \\
		\bottomrule
	\end{tabular} \\[1.5ex]
	Capitals shown are the ones that differ from Roman capitals.
\end{center}

%----------------------------------------------------------------------------------------
%	GLOSSARY
%----------------------------------------------------------------------------------------

% The glossary needs to be compiled on the command line with 'makeglossaries main' from the template directory

\setglossarystyle{listgroup} % Set the style of the glossary (see https://en.wikibooks.org/wiki/LaTeX/Glossary for a reference)
\printglossary[title=Special Terms, toctitle=List of Terms] % Output the glossary, 'title' is the chapter heading for the glossary, toctitle is the table of contents heading

%----------------------------------------------------------------------------------------
%	INDEX
%----------------------------------------------------------------------------------------

% The index needs to be compiled on the command line with 'makeindex main' from the template directory

\printindex % Output the index

%----------------------------------------------------------------------------------------
%	Declaration of Originality
%----------------------------------------------------------------------------------------

\cleardoubleoddpage

\chapter{Declaration of originality}

I hereby confirm that I am the sole author of the written work here enclosed and that I have compiled it in my own words. Parts excepted are corrections of form and content by the supervisor.

\vfill

\makeatletter
\begin{center}
	\begin{tabular}[t]{@{}ll@{}}%
		\textbf{Title of work}: & \@title  \\
		\textbf{Authored by}:   & \@author \\
	\end{tabular}
\end{center}
\makeatother

\vfill

With my signature I confirm that
\begin{itemize}
	\item I have committed none of the forms of plagiarism described in the \enquote{\href{https://www.ethz.ch/content/dam/ethz/main/education/rechtliches-abschluesse/leistungskontrollen/plagiarism-citationetiquette.pdf}{Citation~etiquette}} information sheet.
	\item I have documented all methods, data and processes truthfully.
	\item I have not manipulated any data.
	\item I have mentioned all persons who were significant facilitators of the work.
\end{itemize}

I am aware that the work may be screened electronically for plagiarism.

\signaturefield{}

%----------------------------------------------------------------------------------------

\end{document}
