\setchapterpreamble[u]{%
    \margintoc\hfil%
    \dictum[Oscar Wilde]{You can never be overdressed\\or overeducated.}
}
\chapter{Concepts}
\labch{concepts}


\section{NMR Spectroscopy}
Principles of Pulses and relaxation
rotating reference frame
different variatns: continuous wave, pulse, CIDNP, ...
different goals: structure, chemical shifts, J coupling/electron bonds/...
requirements on homogenity, noise, coil, rf, ...




\section{RF-Engineering}
Q/I Modulation
superheterodyne/homodyne/direct sampling receiver


\section{Concept}
Idea/motivation
given magnet
schema of setup
Design decisions
rf switch
amplifiers

\section{Complete Setup}

\begin{figure*}
    \centering
    \begin{circuitikz}[european]
        \ctikzset{bipoles/amp/width=0.9}
        \draw[nodes={align=center}]
        (0,0) coordinate(mid)

        % TX
        (mid) ++(0,3) node[draw, align=center, minimum height=5.5cm, minimum width=2cm](decoder){Sequence\\Decoder}
        ($(decoder.east)!0.75!(decoder.north east)$) coordinate(decoderi)
        ($(decoder.east)!0.75!(decoder.south east)$) coordinate(decoderq)
        ($(decoder.north west)!0.5!(decoder.south west)$) coordinate(decoderin)
        (decoderin) ++(-1,0) node[left]{Pulse\\Sequence} coordinate(seq) -- (decoderin) node[inputarrow]{}
        (decoderi) node[left]{I} ++(3,0) node[mixer](mi1){}
        (decoderq) node[left]{Q} ++(3,0) node[mixer](mq1){}
        (decoderi) -- (mi1.w) node[inputarrow]{}
        (decoderq) -- (mq1.w) node[inputarrow]{}
        (decoderin -| mi1.s) coordinate (center1)
        (center1) ++(-1,0) node[oscillator](osc1){} -- (center1) -- (mi1.s) node[inputarrow,rotate=90]{}
        (osc1.n) node[above]{NCO}
        (osc1.s) node[below]{\qty{25}{MHz}}
        (center1) to[phaseshifter,>,l=90°] (mq1.n) node[inputarrow,rotate=270]{}
        (center1) ++(2,0) node[adder](add1){}
        (mi1.e) -| (add1.n) node[inputarrow,rotate=270]{}
        (mq1.e) -| (add1.s) node[inputarrow,rotate=90]{}
        (add1.e) to[dac,>] ++(2,0) to[amp,t=PA,l=Power\\Amplifier,>] ++(2,0) coordinate(tx)

        % Circulator
        (mid -| tx) node[circulator](circ){} node[below right]{active\\TX/RX Switch}
        (tx) -| (circ.n) node[inputarrow,rotate=270]{}

        % RX
        (mid) ++(0,-3) coordinate(center_rx)
        (center_rx -| circ) coordinate(rx_out)
        (circ.s) |- (rx_out)
        (rx_out) to[amp,t=\rotatebox{180}{LNA},l=Low Noise\\Amplifier,>] ++(-2,0) to[adc,>] ++(-2,0) -- (rx_out -| add1) coordinate(rx_split)
        (rx_split) ++(0,2) coordinate(rx_upper)
        (rx_split) ++(0,-2) coordinate(rx_lower)
        (rx_upper -| mi1) node[mixer](mi2){}
        (rx_lower -| mq1) node[mixer](mq2){}
        (rx_split) |- (mi2.e) node[inputarrow,rotate=180]{}
        (rx_split) |- (mq2.e) node[inputarrow,rotate=180]{}
        (rx_split -| osc1) node[oscillator](osc2){}
        (osc2.n) node[above]{NCO}
        (osc2.s) node[below]{\qty{25}{MHz}}
        (osc2.e) -| (mi2.s) node[inputarrow,rotate=90]{}
        (osc2.e -| mq2) to[phaseshifter,>,l=90°] (mq2.n) node[inputarrow,rotate=270]{}
        (mi2.w) to[bandpass,l_=CIC,>] (mi2 -| decoderi) node[inputarrow,rotate=180]{} node[left](circ_out_i){I}
        (circ_out_i) -- (mi2 -| seq) node[left]{I} node[inputarrow,rotate=180]{}
        (mq2.w) to[bandpass,l=CIC,>] (mq2 -| decoderq) node[inputarrow,rotate=180]{} node[left](circ_out_q){Q}
        (circ_out_q) -- (mq2 -| seq) node[left]{Q} node[inputarrow,rotate=180]{}

        % Probe
        (circ.e) -- ++(2,0) node[dinantenna](probe){Probe}
        ;

        % Red Pitaya
        \draw[dashed]
        ($(seq)!0.5!(decoderin)$) ++(0,3) coordinate(rectangle_start)
        ($(add1.e)!0.5!(tx)$) coordinate(rp_out)
        (rp_out |- mq2) ++(0,-1) coordinate(rectangle_end)
        (rectangle_start) rectangle (rectangle_end)
        (rectangle_start) node[above right]{Red Pitaya}
        ;
    \end{circuitikz}

    \caption{\captiontitle{The \acrlong{rp} inside the transmit-receive pipeline.} The processing inside the \acrlong{rp} is done digitally on an \acrshort{fpga} before being sent to the high-level control software.}
    \labfig{redpitaya}
\end{figure*}
