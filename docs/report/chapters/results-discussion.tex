\setchapterpreamble[u]{\margintoc\hfil}
\chapter{Results}
\labch{results}

The following sections look at the built spectrometer from a user's point of view. The first \refsec{user-perspective} describes the assembly and operation of the spectrometer. The following \refsec{water-signal} discusses how to measure a signal with water as a simple and approachable example.

\section{The spectrometer from a user's perspective}
\labsec{user-perspective}
The spectrometer was designed with ease of use and reconfigurability in mind. The individual parts are placed on separate boards, connected with standard \acrshort{sma} connectors. Broken parts can thus be easily exchanged. Old already existing parts can be used in conjunction with newly developed ones, facilitating the re-use of hardware and ensuring operation while a broken part is fixed or upgraded.

The software -- while still incomplete -- has the same goals as the hardware. It's written in \gls{python} with extensive documentation, comments throughout the code and accompanying guides to get started\sidenote{Take a look at the official \lstinline{README.md} in the \href{https://gitlab.ethz.ch/mstabel/nmr-spectrometer/-/tree/master/software/spectrometer}{official repository}}. Generally, the code tries to adhere to the ideas presented in \enquote{Uncle Bob's} book \textit{Clean Code} \sidecite{martinCleanCodeHandbook2008}.

\begin{marginfigure}
    \includesvg{images/logo_magnETHical.svg}
    \caption{Logo of the \textit{magnETHical} spectrometer project}
    \labfig{logo-magnethical}
\end{marginfigure}

There are five main parts to the spectrometer as explained in \nrefch{concepts}:
\begin{description}
    \item[The console] (i.e. the RedPitaya) responsible for sending, receiving and processing the \acrshort{rf} signals.
    \item[The power amplifiers] responsible for amplifying the signal generated by the console
    \item[The transmit-receive switch] responsible for switching between sending a signal into the probe from the transmit channel and receiving a signal back from the probe into the receive channel
    \item[The probe] consisting of the probe holder and the probe coil, responsible for emitting and receiving the \acrshort{rf} signal
    \item[The \acrlong{lna}s] responsible for amplifying the weak signal received by the probe before feeding it to the console for processing
\end{description}
The short conceptual overview is reproduced in figure \reffig{overview} for the reader's convenience.

\begin{figure*}[!htb]
    \centering
    \begin{circuitikz}[european]
        \ctikzset{bipoles/amp/width=0.9}
        \draw[nodes={align=center}]
        (0,0) coordinate(mid)

        % RP
        (mid) node[draw, align=center, minimum height=5.5cm, minimum width=2cm](redpitaya){Red\\Pitaya\\SDRlab\\122-16}
        ($(redpitaya.east)!0.75!(redpitaya.north east)$) coordinate(rptx) node[left]{TX\\(Out 1)}
        ($(redpitaya.east)!0.75!(redpitaya.south east)$) coordinate(rprx) node[left]{RX\\(In 1)}

        % TX
        (rptx) to[lowpass,l=SCLF-27+,>] ++(5,0)
        to[amp,t=\acrshort{pa},l=ADL5536\\PHA-202+,>] ++(5,0) coordinate(tx)

        % Circulator
        (mid -| tx) node[circulator,label={left:QPC6324}](circ){}
        (tx) -| (circ.n) node[inputarrow,rotate=270]{}

        % RX
        (rprx -| circ) coordinate(rx)
        (circ.s) |- (rx)
        (rx) to[amp,t=\acrshort{lna},l=PHA-13LN+,>] ++(-2.5,0)
        to[amp,t=\acrshort{lna},l=PHA-13LN+,>] ++(-2.5,0)
        to[amp,t=\acrshort{lna},l=PHA-13LN+,>] ++(-2.5,0)
        to[lowpass,l=SCLF-27+,>] ++(-2.5,0) node[inputarrow,rotate=180]{}

        % Probe
        (circ.e) -- ++(1,0)
        node[dinantenna](probe){}
        node[below=1ex]{3D-printed\\probe holder}
        ;
    \end{circuitikz}

    \caption{\captiontitle{Component overview.} The schematic contains all physical parts of the \acrshort{nmr} spectrometer that need to be connected through \acrshort{sma} cables.}
    \labfig{overview}
\end{figure*}


\section{Measuring a water signal}
\labsec{water-signal}

\todo{Water FID}
\todo{Water Spectrum}
\todo{Wassersignal Spannung (aus vergleich mit Funktionsgeneratorsignal)}

\todo{Rabi Nutation Figure}
\todo{T2 Decay Figure}

\section{Measuring a Toluol signal}
\labsec{toluol-signal}
\todo{Try and measure a Toluol signal}