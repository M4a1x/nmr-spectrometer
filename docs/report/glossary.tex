% used with \gls
\newglossaryentry{computer}{
	name=computer,
	description={is a programmable machine that receives input, stores and manipulates data, and provides output in a useful format}
}
\newglossaryentry{MR}{
	name=Magnetic Resonance,
	description={is a phenomenon where atomic nuclei in a strong constant magnetic field, excited by a second oscillating magnetic field of correct frequency resonate and emit a corresponding electromagnetic signal}
}
\newglossaryentry{NMR}{
	name=Nuclear Magnetic Resonance spectroscopy,
	description={is a spectroscopic technique to observe shifts in magnetic field strengths in atomic nuclei}
}
\newglossaryentry{CW-NMR}{
	name=Continous Wave Nuclear Magnetic Resonance spectroscopy,
	description={is an \gls{NMR} experiment performed by slowly varying the excitation frequency and observing the resonance behaviour of nuclei inside the sample}
}
\newglossaryentry{FT-NMR}{
	name=Fourier Transform Nuclear Magnetic Resonance spectroscopy,
	description={Instead of slowly varying the frequency as in \gls{CW-NMR}, a single pulse of a given length with a fixed frequency is sent. According to Fourier's theory, this \enquote{smears} the frequency and excites nuclei with different resonant frequencies simultaneously}
}

% Glossary entries (used in text with e.g. \acrfull{fpsLabel} or \acrshort{fpsLabel})
\newacronym[longplural={Frames per Second}]{fpsLabel}{FPS}{Frame per Second}
\newacronym[longplural={Tables of Contents}]{tocLabel}{TOC}{Table of Contents}

\newacronym{mr}{MR}{\gls{MR}}
\newacronym{nmr}{NMR}{\gls{NMR}}
\newacronym{ftnmr}{FT-NMR}{\gls{FT-NMR}}
\newacronym{cwnmr}{CW-NMR}{\gls{CW-NMR}}