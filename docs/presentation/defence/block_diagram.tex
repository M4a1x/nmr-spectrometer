\documentclass{standalone}

% Fonts
\usepackage{fontspec}  % Requires LuaLaTeX
\usepackage{mathtools}  % Math
\usepackage[
  warnings-off={
      mathtools-colon,%
      mathtools-overbracket,%
    }
]{unicode-math}% Load after mathtools
\usepackage[tracking]{microtype}
\microtypesetup{protrusion=true,expansion=true}  % classicthesis

\setmainfont[Numbers=Lowercase,Scale=1.04]{TeX Gyre Pagella}
\setsansfont{Merriweather Sans}
\setmonofont[AutoFakeSlant,Scale=MatchLowercase]{inconsolata}
\setmathfont[mathrm=sym,mathit=sym,mathsf=sym,mathbf=sym,mathtt=sym,NFSSFamily=tgpl]{TeX Gyre Pagella Math}
\newfontfamily{\unitnumberfont}[Numbers=Uppercase]{Merriweather Sans}%TeX Gyre Pagella}  % Upright numbers for units in siunitx
\renewcommand{\familydefault}{\sfdefault}

\usepackage{pgfplots}		% Also loads Tikz
\pgfplotsset{compat=1.18}       % Set compatibility version. Can be updated as required
\usepackage{pgfplotstable}	% For advanced table calcs, e.g. 'y expr' to mathematically process input data:
\usepackage[edges]{forest}	% Comprehensive trees
\usepackage{tikz-3dplot}

% Only use libraries as needed to keep compilation times low
\usetikzlibrary{
  positioning,	% Relative positioning etc.
  calc,			% Calculate distances, coordinates etc.
  shapes, 		% For cross out
  backgrounds,	% Draw on background layer
  fit,			% Fit new node around existing coordinates
  decorations,
  arrows,
  arrows.spaced,
  intersections,
  trees,
  % circuits.ee.IEC,% Electrical engineering circuits lib
  patterns,
  3d,
  tikzmark,		% Marks/coordinates at arbitrary positions
}
\usepgfplotslibrary{
  colorbrewer,
  units,
  dateplot,
  fillbetween,
  groupplots,
}
\usepackage[european, siunitx]{circuitikz}

% Print a contour around letters, e.g. black text with a white border on some noisy
% background, so that text remains legible.
% NOT COMPATIBLE WITH XELATEX! Requires pdflatex or lualatex (our case)
\usepackage[outline]{contour}
\contourlength{0.12em}

\newcommand*{\ctrw}[1]{% Black text, white contour
  \hypersetup{hidelinks}% Hide links, which might be colored
  \contour{white}{\textcolor{black}{#1}}%
}%

\newcommand*{\ctrb}[1]{% White text, black contour
  \hypersetup{hidelinks}% Hide links, which might be colored
  \contour{black}{\textcolor{white}{#1}}%
}


\begin{document}
\begin{circuitikz}
    \ctikzset{bipoles/amp/width=0.9}
    \draw[nodes={align=center}]
    (0,0) coordinate(mid)


    % RP
    (mid) node[draw, minimum height=5.5cm, minimum width=2.5cm,label=Console](redpitaya){}
    (redpitaya.west) coordinate(inp) node[right]{Input}
    ($(redpitaya.east)!0.75!(redpitaya.north east)$) coordinate(rptx) node[left]{Transmit}
    ($(redpitaya.east)!0.75!(redpitaya.south east)$) coordinate(rprx) node[left]{Receive}

    % Computer
    (-3,0) node(seq){Sequence}
    (seq.east) -- (redpitaya.west) node[inputarrow]{} coordinate(console)

    % TX
    (rptx) to[lowpass,l=Lowpass,>] ++(3,0)
    to[amp,l=Amplifier,>] ++(3,0) coordinate(tx)

    % Circulator
    (mid -| tx) node[circulator,label={left:Switch}](circ){}
    (tx) -| (circ.n) node[inputarrow,rotate=270]{}

    % RX
    (rprx -| circ) coordinate(rx)
    (circ.s) |- (rx)
    (rx) to[amp,l=Amplifier,>] ++(-3,0)
    to[lowpass,l=Lowpass,>] ++(-3,0) node[inputarrow,rotate=180]{}

    % Probe
    (circ.e) -- ++(1,0)
    node[dinantenna](probe){}
    node[below=1ex]{Coil}
    ;
\end{circuitikz}
\end{document}