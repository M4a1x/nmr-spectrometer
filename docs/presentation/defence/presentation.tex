\documentclass{ethpresentation}
\usepackage{multimedia}
% \setbeameroption{show notes on second screen} % use e.g. pympress to view

% Simple handout with just the slides
% "article" mode is better, but this might be useful in a pinch?
% \documentclass[handout]{ethpresentation}
% \usepackage{pgfpages}
% \pgfpagesuselayout{6 on 1}[a4paper,border shrink=5mm]

% Footnotes for sources without number
\let\svthefootnote\thefootnote
\NewDocumentCommand{\freefootnote}{m}{
  \let\thefootnote\relax%
  \footnotetext{#1}%
  \let\thefootnote\svthefootnote%
}

% Regular footnotes use letters
\renewcommand*{\thefootnote}{\fnsymbol{footnote}}

% Logo
\NewDocumentCommand{\magnethical}{}{\textnormal{\textsl{\textmd{\textsl{magn\textbf{ETH}ical}}}}}

% Notes:
% - Include challenges faced and overcome, so it's mentioned
% - Include lessons learned


%%%%%%%%%%%%%%%%%%%%%%%%%%%%%%%%%%%%%%%%%%%%%%%%%%%%%%%%%%%%%%%%%%%%%%%%%%%%%%%
% Metadata
%%%%%%%%%%%%%%%%%%%%%%%%%%%%%%%%%%%%%%%%%%%%%%%%%%%%%%%%%%%%%%%%%%%%%%%%%%%%%%%

\title{\Huge\magnethical}
\subtitle{Building a 25\,MHz NMR Spectrometer}
\date{\today}
\author{Maximilian Stabel}
\institute{ETH Zürich}


%%%%%%%%%%%%%%%%%%%%%%%%%%%%%%%%%%%%%%%%%%%%%%%%%%%%%%%%%%%%%%%%%%%%%%%%%%%%%%%
% Content
%%%%%%%%%%%%%%%%%%%%%%%%%%%%%%%%%%%%%%%%%%%%%%%%%%%%%%%%%%%%%%%%%%%%%%%%%%%%%%%

\begin{document}
\maketitle % show when people are walking in

% Attention Getter
\begin{frame}
  \centering
  \emph{\enquote{What I cannot create, I do not understand}}

  \vspace{\baselineskip}

  \hfill{}---Richard Feynman
  \note[item]{In the spirit of Richard Feynman we will together build an NMR machine and make sure we truly understand.}
  \note[item]{This presentation should hopefully be understood by everyone here, but please interrupt me if something isn't clear.}
  \note[item]{I hope everyone learns something new today}
  \note[item]{Even if it is only a newfound or reawoken appreciation for the wonders of magnetic resonance}
  \note[item]{Transition: But before diving in --- you might ask why we care about building an NMR? Just use it?}
\end{frame}

% Need / Motivation
\section*{Why?}
\note{Reminder: What is NMR spectrometer}

\begin{frame}{Nuclear Magnetic Resonance}
  \begin{itemize}
    \item Nuclei absorb radio waves at a certain frequency when inside a magnetic field
    \item The nuclei emit radio waves at that same frequency when excited this way
    \item The frequency depends on the magnetic field and the surrounding electrons/atoms
    \item NMR can be used to determine the structure and composition of a sample
  \end{itemize}
\end{frame}


\note{
  \begin{itemize}
    \item You: Understanding for own experiments
    \item The better we know the better we can use
    \item Push NMR development --- better machines
    \item Transition: if not about you personally --- more globally: applications
  \end{itemize}
}

\begin{frame}{NMR is used across various fields}
  \note[item]{Some of you already know, but here are some reasons why NMR is useful}
  \begin{itemize}[<+->]
    \item Research (Structure Analysis, Drug Discovery, \ldots)
    \item Medicine (Imaging, Diagnosis, \ldots)
    \item Industry (Process Control, Drug screening, \dots)
    \item Education (Quantum Mechanics, Quantum Computing, \ldots)
  \end{itemize}
\end{frame}

\begin{frame}{There is not a lot of NMR research in the Global South}
  \centering
  \includegraphics[height=0.8\textheight]{images/nmr-affiliations-per-million-people_naturalbreaks.pdf}

  \(\frac{\text{NMR publications}}{\text{million people}}\)
  \note[item]{Only publications, doesn't include use of NMR!}
\end{frame}

% Task / Main Message
% One sentence to remember if they remember only one
\begin{frame}[standout]
  \centering
  Build an accessible\\NMR spectrometer
\end{frame}

% Preview
% Give short overview of main points to come
% DONT just list ("I will talk about ..., then about ...)
% Integrate with main message, include audience ("we")
% Introduction to NMR
% Will talk about the parts of the spectrometer
% Do a short demonstration..
% Show table of contents, without anything above (introduction, ...) and without conclusion/review/close
\begin{frame}{Preview}
  \tableofcontents
\end{frame}

% Main Body
% Remember to do transitions!
\section{The parts}
\begin{frame}{Our goal is to build an accessible NMR spectrometer}
  \centering
  \begin{circuitikz}
    \ctikzset{bipoles/amp/width=0.9}
    \draw[nodes={align=center}]
    (0,0) coordinate(mid)


    % RP
    (mid) node[draw, minimum height=5.5cm, minimum width=2.5cm,label=Console](redpitaya){}
    (redpitaya.west) coordinate(inp) node[right]{Input}
    ($(redpitaya.east)!0.75!(redpitaya.north east)$) coordinate(rptx) node[left]{Transmit}
    ($(redpitaya.east)!0.75!(redpitaya.south east)$) coordinate(rprx) node[left]{Receive}

    % Computer
    (-3,0) node(seq){Sequence}
    (seq.east) -- (redpitaya.west) node[inputarrow]{} coordinate(console)

    % TX
    (rptx) to[lowpass,l=Lowpass,>] ++(3,0)
    to[amp,l=Amplifier,>] ++(3,0) coordinate(tx)

    % Circulator
    (mid -| tx) node[circulator,label={left:Switch}](circ){}
    (tx) -| (circ.n) node[inputarrow,rotate=270]{}

    % RX
    (rprx -| circ) coordinate(rx)
    (circ.s) |- (rx)
    (rx) to[amp,l=Amplifier,>] ++(-3,0)
    to[lowpass,l=Lowpass,>] ++(-3,0) node[inputarrow,rotate=180]{}

    % Probe
    (circ.e) -- ++(1,0)
    node[dinantenna](probe){}
    node[below=1ex]{Coil}
    ;
  \end{circuitikz}
  \note[item]{Go through the parts clockwise}
\end{frame}

\begin{frame}{The console\\is a ready-made FPGA board\footnote{Red Pitaya SDRlab 122-16}}
  \centering
  \includegraphics[height=0.8\textheight]{rp122-16.png}
  \note[item]{FPGA == programmable hardware, very fast}
  \note[item]{oversampling}
\end{frame}
% out
% TODO: Oversampling, averaging cic
% MaRCoS

\begin{frame}{An amplifier is basically just a transistor}
  \centering
  \begin{columns}
    \begin{column}{0.5\textwidth}
      \begin{itemize}
        \item Transistor:\\voltage-controlled current source
        \item \(
              \begin{aligned}[t]
                \text{higher voltage} & \rightarrow{} \text{higher current}          \\
                                      & \rightarrow{} \text{higher voltage } V_{R_C} \\
                                      & \rightarrow{} \text{lower voltage } V_{CE}   \\
                                      & \rightarrow{} \text{\ang{180} phase shift}
              \end{aligned}
              \)
      \end{itemize}

    \end{column}
    \begin{column}{0.5\textwidth}


      \begin{circuitikz}
        \draw[nodes={align=center}]
        (0,0) node[npn](Q){}
        (Q.B) to[short,-o] ++(-0.5,0) node[left]{\(V_{in}\)}
        (Q.C) to[R,l=\(R_C\),v<=\(V_{R_C}\)] ++(0,2) node[vcc]{\(V_{CC}\)}
        (Q.C) to[short,*-o] ++(1,0) node[right]{\(V_{out}\)}
        (Q.E) node[ground]{}
        ([shift={(0.1,0)}]Q.E) to[open,v_<=\(V_{CE}\)] ([shift={(0.1,0)}]Q.C)
        ;
      \end{circuitikz}
    \end{column}
  \end{columns}
  \note[item]{We want cheap, so using a simple is the obvious first approach}
  \note[item]{Unfortunately there's a lot to do:%
    \begin{itemize}
      \item Input/Output Impedance Matching
      \item Bias Tee
      \item DC coupling
      \item stability calculations
      \item feedback
      \item temperature compensation (current feedback)
    \end{itemize}
  }
  \note[item]{A complete amplifier is quite expensive}
  \note[item]{Solution: Use monolithic (integrated) amplifier}
  \note[item]{Take care of heat dissipation (Class-A)}
\end{frame}

\begin{frame}{The power amplifier has two stages}
  \centering
  \includegraphics[width=0.9\textwidth]{poweramp.png}
\end{frame}
% db repetition
% schwarzer block von mini-circuits

\begin{frame}{The passive approach leaked too much power}
  \centering

  \begin{circuitikz}
    \ctikzset{
      diode=empty,
      diodes/scale=0.8
    }
    \draw[nodes={align=center}]

    % tx diodes
    (0,0) node[left]{Transmit} to[short,o-*] ++(1,0) coordinate(tx)
    (tx) -- ++(0, 0.5) to[D] ++(1.5,0) -- ++(0,-0.5)
    (tx) -- ++(0,-0.5) to[D,invert] ++(1.5,0) -- ++(0, 0.5)
    to[short,*-] ++(1,0) coordinate(mid)

    % A/B
    (mid) node[below](A){A}
    (mid) to[short,*-o] ++(0,2) node[above]{Probe}
    (mid) -- ++(3,0) coordinate(b)
    (b) node[above](B){B}

    % rx diodes
    (b) to[short,-*] ++(0,-0.5) coordinate(rx)
    (rx) -- ++(-0.5,0) to[D] ++(0,-1.5) -- ++(0.5,0)
    (rx) -- ++(0.5,0) to[D,invert] ++(0,-1.5) -- ++(-0.5,0)
    to[short,*-] ++(0,-0.5) node[ground]{}

    (b) to[short, *-o] ++(1,0)
    node[right]{Receive}
    ;
    \draw[<->] (A.east |- B.west) -- (B.west);
    \draw ($(A.east|-B.west)!0.5!(B.west)$) node[above]{\(\frac{\lambda}{4}\)};
  \end{circuitikz}

  \note[item]{So called "video feedthrough"}
  \note[item]{Especially noise during reception phase, leaking through turned off amplifier}
  \note[item]{"Traditional" passive design by Lowe and Tarr}
  \note[item]{Leads to distortion of low-power pulses}
  \note[item]{Same design can be used with PIN-Diodes (effectively current-controlled resistor)}
  \note[item]{But PIN Diodes often need higher frequencies (mid MHz), size of intrinsic semiconductor}
\end{frame}

\begin{frame}{A transmission line transforms impedance}
  \begin{columns}
    \begin{column}{0.6\textwidth}
      \centering
      \only<2->{
        \movie[width=0.8\textwidth]{%
          \includegraphics[width=0.8\textwidth]{transmission_line/transmission_line.png}
        }{transmission_line.gif}
      }
    \end{column}
    \begin{column}{0.4\textwidth}
      \vspace*{-0.5cm}

      \begin{circuitikz}
        \draw (0,0) to[short,o-] ++(-3,0) to[vsource,v=\(V_0\)] ++(0,2) to[R,l=\(Z_0\),-o] ++(3,0);
      \end{circuitikz}

      \vspace*{0.75cm}

      \hrule{}

      \vspace*{0.25cm}

      \begin{circuitikz}
        \draw (0,0) to[short,o-] ++(-3,0) to[vsource,v=\(V_0\)] ++(0,2) to[R,l=\(Z_0\),-o] ++(3,0);
        \draw (0,0) to[short,o-] ++(1,0) -- ++(0,2) to[short,-o] ++(-1,0);
      \end{circuitikz}
    \end{column}
  \end{columns}
\end{frame}

\begin{frame}{We use a transistor-based active switch}
  \centering
  \includegraphics[height=0.9\textheight]{tr_switch.png}
  \note[item]{Active}
  \note[item]{Isolation: \qty{60}{\deci\bell}}
  \note[item]{Silicon-on-insulator (not pHEMT GaAs) i.e. FET tech, not PIN-Diode}
  \note[item]{PIN-Diode switch also possible, but
    \begin{itemize}
      \item usually higher leakage
      \item slower switching
      \item harder to integrate on a chip
      \item but higher power capabilities
    \end{itemize}
  }
\end{frame}

% TODO: Add how it works
\begin{frame}{The low-noise amplifier had instability issues}
  \centering
  \includegraphics[width=0.9\textwidth]{preamp.png}
  \note[item]{Low-noise }
  \note[item]{Feedback loop --- stray capacitance}
  \note[item]{Solution: Smaller housing, shorter loop}
\end{frame}

\begin{frame}{The probe}
  \centering
  \includegraphics[height=0.8\textheight]{probe.jpg}
  \note[item]{Many turns --- high inductance --- low capacitance --- sensitive to stray capacitance}
\end{frame}

\begin{frame}{A 32-channel current supply is designed but untested}
  \centering
  \includegraphics[height=0.8\textheight]{32-channel_current_source.png}
\end{frame}

\section{The complete setup}

% Add picture of complete setup

\begin{frame}{Our NMR is affordable \ldots}
  \begin{table}
    \begin{tabular}{@{}
        l
        S[table-format=7.0,table-align-text-pre=false]
        S[table-format=3.2,table-align-text-pre=false]
        S[table-format=5.2,table-align-text-pre=false]
        @{}}
      \toprule
                          & {\qty{600}{\mega\hertz}\footnote{estimated costs}} & {mini-circuits} & {\magnethical}    \\
      \midrule
      Power Amplifier     & 50000                                              & 323.49          & 36.01             \\
      Switch              & {-}                                                & 82.06           & 20.05             \\
      Probe               & 100000                                             & {-}             & {\approx} 15.00   \\
      Low-Noise Amplifier & 50000                                              & 409.38          & 73.11             \\
      Shim Driver         & {-}                                                & {-}             & 257.08            \\
      Console             & 200000                                             & {-}             & 662.53            \\
      Magnet              & 1000000                                            & {-}             & {\approx} 9000.00 \\
      \bottomrule
      \textbf{Sum}        &                                                    &                 & \textbf{10142.80} \\
    \end{tabular}
  \end{table}
  % \freefootnote{Prices incl. VAT [CHF]}
  % \textsuperscript{\ref{foot}}
  %\footnote{Integrated in Console\label{foot}}
  % Add mini-circuits prices
  % Add superconducting prices? see notes
\end{frame}

\begin{frame}{\ldots{} and competitive}
  \begin{table}
    \begin{tabular}{@{} lrrr @{}}
      \toprule
                                     & Superconducting                        & Benchtop                            & \magnethical{}                                             \\
      \midrule
      Price [k\,CHF]                 & \numrange[range-phrase=--]{200}{18000} & \numrange[range-phrase=--]{50}{150} & \approx\num{10}                                            \\
      Frequency [\unit{\mega\hertz}] & \numrange[range-phrase=--]{300}{1200}  & \numrange[range-phrase=--]{40}{125} & \num{25}                                                   \\
      Resolution [\unit{\hertz}]     & \approx\num{0.2}                       & \numrange[range-phrase=--]{0.2}{1}  & \approx \num{2.5}/\num{50}\footnote[2]{with/without shims} \\
      Weight [\unit{\kilo\gram}]     & \numrange[range-phrase=--]{600}{15000} & \numrange[range-phrase=--]{25}{150} & \approx\num{5}                                             \\
      % \bottomrule
    \end{tabular}
  \end{table}
  \freefootnote{For 5mm standard NMR tubes}
\end{frame}

% Results
\section{Measurement Results}

\begin{frame}{Simple Pulse Sequence}
  \centering
  \includesvg[width=0.5\textwidth]{simple_pulse_sequence.svg}
\end{frame}

\begin{frame}{We can already see a water FID}
  \centering
  \includegraphics[width=0.9\textwidth]{images/fid_sine_fit.pdf}
  \note[item]{Add T2* of fit here!}
  \note[item]{Outliers at beginning are due to CIC filters}
\end{frame}

\begin{frame}{\ldots and do a Fourier transform}
  \centering
  \begin{tikzpicture}
    \node (fft)
    {
      \includegraphics[width=0.9\textwidth]{images/fft_fit.pdf}
    };
    \begin{scope}[overlay]
      \node (structure) at (fft.north)
      [
        anchor=center,
        xshift=0.65cm,
        yshift=0cm
      ]
      {
        \includesvg[height=0.1\textheight]{images/h2o.svg}
      };
      \path[draw=grey,dashed,->] ([yshift=1mm,xshift=-3mm]structure.south east) to ([xshift=0.7cm,yshift=2.5cm]fft.center);
      \path[draw=grey,dashed,->] ([yshift=1mm,xshift=3mm]structure.south west) to ([xshift=0.6cm,yshift=2.5cm]fft.center);
    \end{scope}
  \end{tikzpicture}
  \note{Ask neighbours about lineshape, why not lorentz?}
\end{frame}

\begin{frame}{Toluene also has a visible signal}
  \centering
  \includegraphics[width=0.9\textwidth]{images/fid_toluene.pdf}
\end{frame}

\begin{frame}{We can even see the chemical shifts\\of the Toluene peaks!}
  \centering
  \begin{tikzpicture}
    \node (fft)
    {
      \includegraphics[width=0.9\textwidth]{images/fft_toluene.pdf}
    };
    \begin{scope}[overlay]
      \node (structure) at (fft.north)
      [
        anchor=north west,
        xshift=0mm,
        yshift=1.5cm
      ]
      {
        \includesvg[height=0.15\textheight]{images/toluene.svg}
      };
      \path[draw=grey,dashed,->] ([yshift=5mm,xshift=-4mm]structure.south east) to ([yshift=1.5cm,xshift=1.75cm]fft.center);
      \path[draw=grey,dashed,->] ([yshift=2mm,xshift=2mm]structure.south west) to ([yshift=2.75cm]fft.center);
    \end{scope}
  \end{tikzpicture}
  \note{It's a full tube of toluene, not a solution}
  \note[item]{Reasons for sidepeak: no apodization (truncation of FID), no shimming, inhomogeneities/not centred}
\end{frame}

\begin{frame}{Rabi nutation (pulse calibration) of water}
  \centering
  \includegraphics[width=0.9\textwidth]{rabi_nutation_fit.pdf}
  \note[item]{\(T_{period} = \qty{32}{\micro\second}\)}
  \note[item]{\(T_{\frac{\pi}{2}} = \qty{8}{\micro\second}\)}
  % TODO: Add arrow to pi/2 pulse with pi/2 pulse length (8us?)
\end{frame}

\begin{frame}{Spin Echo Sequence}
  \centering
  \includesvg[width=0.9\textwidth]{spin_echo_sequence.svg}
\end{frame}

\begin{frame}{Spin Echo Animation}
  \begin{center}
    \movie[width=8.7cm,height=6.5cm]{\includegraphics[height=6.5cm,width=8.7cm]{hahn_echo_cc_gwm.png}}{./hahn_echo_cc_gwm.gif}
  \end{center}
  \freefootnote{\href{https://commons.wikimedia.org/wiki/File:GWM_HahnEcho.gif}{Gavin Morley --- CC BY-SA 3.0}}
\end{frame}

\begin{frame}{Spin Echo Measurement}
  \centering
  \includegraphics[width=0.9\textwidth]{spin_echo_avg.pdf}
\end{frame}

\begin{frame}{\(T_2\) Decay Animation}
  \begin{center}
    \movie[width=8.7cm,height=6.5cm]{\includegraphics[height=6.5cm,width=8.7cm]{hahn_echo_cc_gwm.png}}{./hahn_echo_decay_cc_gwm.gif}
  \end{center}
  \freefootnote{\href{https://commons.wikimedia.org/wiki/File:GWM_HahnEchoDecay.gif}{Gavin Morley --- CC BY-SA 3.0}}
\end{frame}

\begin{frame}{\(T_2\) decay of water}
  \centering
  \includegraphics[width=0.9\textwidth]{t2_decay_fit.pdf}
  \note[item]{\(T_2 = \qty{190}{\milli\second}\)}  % Add to slide
\end{frame}

% Demonstration! Measure FID of water
% TODO: Backup slides with notebook screenshots of measurements
\begin{frame}[standout]
  Demo time!
\end{frame}

% Review
% We've understood why NMR is desirable
% We've gone through the concept of what it is
% We've step by step built the spectrometer from console, tx, rx, switch and probe
% We've now seen how a the software works and performed a measurement
% I hope I could convince you that the whole setup is working
% Now the individual parts can be improved individually

\begin{frame}{The setup is working}
  % Now the individual parts can be improved further
  % Outlook!
  % Shim driver, include same as others.. challenges etc.
\end{frame}

% Conclusion
% Restate main message
% complement with other interpretations 

% Close
\begin{frame}{}
  \begin{center}
    \note[item]{Circling back to the beginning, I would like to end with a quote by E.M. Purcell}
    \emph{
      \enquote{I have not yet lost that sense of wonder, and of delight, that this delicate motion should reside in all ordinary things around us, revealing itself only to him who looks for it.}

      \vspace{2\baselineskip}

      \enquote{There the snow lay around my doorstep --- great heaps of protons quietly precessing in the Earth's magnetic field.}
      % In that on the first snowy, silent night this winter you'll think of the quiet precessions in the earth's magnetic field.
    }
  \end{center}
  \hfill{}--- E.M.\ Purcell
  \note[item]{I wouldn't have thought I would get a glimpse of this wonder that Purcell describes when starting my thesis here, but I'm glad I did.}
  \note[item]{And I hope none of you have lost it yet}
\end{frame}

% End: Show during questions
\begin{frame}{Thank you!}
  \begin{columns}
    \begin{column}{0.5\textwidth}
      \centering
      \includesvg[width=0.5\textwidth]{./images/logo_magnETHical.svg}
    \end{column}
    \begin{column}{0.5\textwidth}
      \centering
      Find everything on \\ \vspace*{\baselineskip}

      \includesvg[width=0.5\textwidth]{./images/gitlab_qrcode.svg}

      \url{https://gitlab.ethz.ch/mstabel/nmr-spectrometer}
    \end{column}
  \end{columns}
\end{frame}


%%%%%%%%%%%%%%%%%%%%%%%%%%%%%%%%%%%%%%%%%%%%%%%%%%%%%%%%%%%%%%%%%%%%%%%%%%%%%%%
% Appendix / Backup Slides
%%%%%%%%%%%%%%%%%%%%%%%%%%%%%%%%%%%%%%%%%%%%%%%%%%%%%%%%%%%%%%%%%%%%%%%%%%%%%%%

\appendix

\begin{frame}[standout]
  Backup
\end{frame}

\end{document}